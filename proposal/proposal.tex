\documentclass[12pt]{article}

\usepackage{fullpage}
\usepackage{fancybox}
\usepackage{graphicx}
\usepackage{amssymb}
\usepackage{amsmath}
\usepackage{enumerate}
\usepackage{parskip}
\usepackage{hyperref} 
\usepackage{paralist}


%%%%%%%%%%%%%% Capsule %%%%%%%%%%%%%%%%%%%%%%%%%%%%%%%%%%%%%%%%%%%
\newcommand{\capsule}[2]{\vspace{0.5em}
  \shadowbox{%
    \begin{minipage}{.90\linewidth}%
      \textbf{#1:}~#2%
    \end{minipage}}
  \vspace{0.5em} }
%%%%%%%%%%%%%%%%%%%%%%%%%%%%%%%%%%%%%%%%%%%%%%%%%%%%%%%%%%%%%%%%%%

\newcounter{ques}
\newenvironment{question}{\stepcounter{ques}{\noindent\bf Question \arabic{ques}:}}{\vspace{5mm}}

\begin{document} 

\begin{center} \Large\bf
RL for Tower Defense with Evolutionary Towers\\
\end{center} 

\begin{center}
Group 12 \\
Andrew Wallace - 101210291 - andrewwallace3@cmail.carleton.ca\\
Mohammad Rehman - 101220514 - mohammadrehman@cmail.carleton.ca \\
Manal Hassan - 101263813 - manalhassa@cmail.carleton.ca\\
Derrick Zhang - 101232374 - derrickzhang@cmail.carleton.ca
\end{center}

\section*{Problem Statement}
We aim to develop and train a reinforcement learning agent to play an evolving tower defense game where strategic resource allocation and adaptive tower placement are critical for success. Unlike traditional tower defense games with manual upgrades, our environment features towers that automatically evolve and strengthen based on the number of enemies they eliminate, creating a feedback loop between decisions and long-term strategy when it comes to positioning.\par

The agent must learn to place two types of towers: single target and area-of-effect. The environment will be grid-based with limited budget constraints per wave. As enemy waves progress with increasing health, the agent must balance immediate defensive needs with long-term tower development, deciding not only where to place towers but also which towers to prioritize for enemy elimination to maximize evolutionary potential.\par

This problem presents a few challenges: immediate survival of ramping waves against long-term tower development, spatial reasoning for optimal tower placement on a constrained grid, resource allocation under budget constraints.
The evolving tower mechanism is a novel aspect that creates a unique issue for reinforcement learning algorithms that must learn both short-term and long-term strategies. \par


\section*{Feasibility}

Reinforcement learning (RL) is well-suited for this problem because the environment is stochastic, sequential, and dynamic, with an agent's actions influencing future states. \par 

As enemy waves increase in difficulty, the agent must adapt in terms of placing towers, what type of towers to place, and positional strategy for tower growth. The sequential relationship of agent action and future success is a delayed reward optimization problem that RL methods, such as Q-learning, specialize in, making RL a good fit for this problem [2]. Furthermore, the presence of a large and dynamic state space (tower types, budget, enemy health, and a gridbased environment) creates a large and dynamic state space, making traditional machine learning techniques infeasible, further supporting the use of RL. [3] \par

On the other hand, RL alone may be infeasible. Similar work in a grid-based tower defense game has shown, “A fully-fledged RL agent, trained to select one of the units and place it wherever desired, is hardly feasible in this case” [4]. As such, techniques beyond reinforcement learning alone may be considered in this project. \par

Lastly, the evolving tower mechanism further complicates the balance of immediate survival and long-term tower development, which is a direct RL use case. \par


\section*{Milestones}

\textbf{Initial Environment Setup (Sept 29--Oct 13)}: Implement basic gym environment; submit progress report (Oct 15). 

\textbf{Final Environment Implementation (Oct 14--Oct 27)}: Complete custom gym environment; test with random agent; submit demo video (10 min). 

\textbf{Training and Initial Results (Oct 28--Nov 10)}: Research RL algorithms; begin implementation; submit progress report (Oct 30). 

\textbf{Final Algorithm Implementation (Nov 11--Nov 24)}: Implement working RL algorithms; gather initial results; submit progress report (Nov 15). 

\textbf{Result Demo (Nov 25--Dec 1)}: Collect results; perform algorithm comparison; submit progress report (Nov 30); submit demo video (Dec 1). 

\textbf{Final Report Writing (Dec 2--Dec 7)}: Each member writes a section; compile and review; submit final report (Dec 8).


\vspace{0.5em}

\section*{References}

[1] Sindhu Padakandla. 2021. A Survey of Reinforcement Learning Algorithms for Dynamically Varying Environments. ACM Comput. Surv. 54, 6, Article 127 (July 2021), 25 pages.
https://doi.org/10.1145/3459991 \par

[2] J. Clifton, E. Laber. (2020). Q-Learning: Theory and Applications. [ONLINE]. Available: https://www.annualreviews.org/content/journals/10.1146/annurev-statistics-031219-041220 \par

[3] Dias, A., Foleiss, J., Lopes, R.P. (2022). Reinforcement Learning in Tower Defense. In: Barbedo, I., Barroso, B., Legerén, B., Roque, L., Sousa, J.P. (eds) Videogame Sciences and Arts. VJ 2020. Communications in Computer and Information Science, vol 1531. Springer, Cham. https://doi.org/10.1007/978-3-030-95305-8\_10 \par 

[4] J. Bergdahl, A. Sestini and L. Gisslén, "Reinforcement Learning for High-Level Strategic Control in Tower Defense Games," 2024 IEEE Conference on Games (CoG), Milan, Italy, 2024, pp. 1-8, doi: 10.1109/CoG60054.2024.10645621 \par

[5] Gym library. https://gym.openai.com/. Accessed September 20 2025

\end{document} 